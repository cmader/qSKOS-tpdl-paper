%!TEX root = main.tex

\section{Analysis of Existing SKOS Vocabularies}\label{sec:analysis}

To learn about possible quality issues in real-world vocabularies, we implemented the previously described quality checking function in the qSKOS tool and applied each function on a set of existing SKOS vocabularies. 

\subsection{The qSKOS Quality Checking Tool}

The open-source qSKOS\footnote{\url{https://github.com/cmader/qSKOS/}} quality assessment tool can be used to find possible quality issues a given SKOS vocabulary. Users can run the tool by passing a vocabulary and selecting the quality checking functions they want to perform. As a result, they obtain detailed reports listing possibly affected resources alongside with the potential cause of the issue. qSKOS is implemented in Java and can be used as standalone command-line tool or API in any other application. Its design is open for the introduction of additional quality functions. In our analysis, we used qSKOS version 0.2., which can be downloaded from: \url{https://github.com/cmader/qSKOS/releases/qSKOS-0.2.tar.gz}.

\subsection{Vocabulary Data Set}

We used Sindice and available listings\footnote{\url{http://www.w3.org/2001/sw/wiki/SKOS/Datasets}} to learn about existing SKOS vocabularies. Table~\ref{tab:vocabs} lists our representative vocabulary selection and shows that the vocabularies differ in size and application domain: the \emph{Gemeenschappelijke Thesaurus Audiovisuele Archieven (GTAA)} is vocabulary from the media domain, \emph{Eurovoc} and the \emph{Integrated Public Sector Vocabularies (PSV)} cover public sector concepts, the \emph{Medical Subject Headings (MeSH)} and \emph{Peroxisome Knowledge Base (PXV)} are from the (bio-)medical domain, the \emph{Geonames Ontology} covers locations, the \emph{Thesaurus for Economics (STW)} and \emph{North American Industry Classification System (NAICS)} are vocabularies from the economics and business area and the \emph{DBPedia Categories} reflect Wikipedia's user-generated categorization system. We also include the \emph{Meketre} vocabulary, which defines Egyptology-related concepts, and the non-public \emph{Austrian Armed Forces Thesaurus (LVAk)}, which we generated from CSV files provided by the vocabulary maintainers. A zip-file containing all 14 public vocabulariesx we used in our analysis, can be downloaded from: \url{http://www.github.com/cmader/???}.

\begin{table}
\label{tab:vocabs}
\caption{Analyzed SKOS vocabularies}
    
\begin{center}
\resizebox{\textwidth}{!} {
\setlength{\extrarowheight}{5pt}

\begin{tabular}{p{6cm}ccccccccc}

\textbf{Vocabulary} & \rotatebox{90}{\textbf{Abbreviation}} & \rotatebox{90}{\textbf{Version/Year}} & \rotatebox{90}{\textbf{Concepts}} & \rotatebox{90}{\textbf{Auth. Concepts}} & \rotatebox{90}{\textbf{Labels}} & \rotatebox{90}{\textbf{Semantic Rel.}} & \rotatebox{90}{\textbf{Aggregation Rel.}} & \rotatebox{90}{\textbf{Concept Schemes}} \\
\toprule
Agricultural Thesaurus & \textbf{AGROVOC} & 1.3 & 32,035 & 32,035 & 620,629 & 65,934 & 32,085 & 1 \\
\hline
DBpedia Categories & \textbf{DBpedia} & 3.7 & 743,410 & 743,410 & 740,352 & 1,490,316 & 0 & 0 \\
\hline
The EU's multilingual thesaurus & \textbf{Eurovoc} & 5.0 & 6,797 & 6,797 & 457,788 & 18,491 & 15,512 & 128 \\
\hline
Geonames Ontology & \textbf{Geonames} & 2.2.1 & 671 & 671 & 671 & 0 & 671 & 9 \\
\hline
Gemeenschappelijke Thesaurus Audiovisuele Archieven & \textbf{GTAA} & 2010/08/25 & 171,991 & 171,991 & 178,776 & 50,892 & 343,980 & 9 \\
\hline
Integrated Public Sector Vocabulary & \textbf{IPSV} & 2.00 & 4,732 & 3,080 & 7,945 & 13,843 & 4,483 & 3 \\
\hline
Library of Congress Subject Headings & \textbf{LCSH} & 2011/08/09 & 459,182 & 407,908 & 746,076 & 595,754 & 815,816 & 19 \\
\hline
Austrian Armed Forces Thesaurus & \textbf{LVAk} & n/a & 13,411 & 13,411 & 17,250 & 16,346 & 0 & 0 \\
\hline
Middle Kingdom tombs of Ancient Egypt Thesaurus & \textbf{Meketre} & 2011/07/07 & 422 & 422 & 569 & 1,698 & 6 & 2 \\
\hline
Medical Subject Headings & \textbf{MeSH} & 2006 & 24,626 & 24,626 & 150,617 & 38,858 & 0 & 0 \\
\hline
North American Industry Classification System & \textbf{NAICS} & 2012 & 4,175 & 2,213 & 0 & 8,684 & 2,235 & 1 \\
\hline
New York Times People & \textbf{NYTP} & 2010/06/22 & 4,979 & 4,979 & 4,979 & 0 & 4,979 & 1 \\
\hline
University of Southampton Pressinfo & \textbf{Pressinfo} & 2011/02/24 & 1,125 & 1,125 & 0 & 0 & 0 & 0 \\
\hline
Peroxisome Knowledge Base & \textbf{PXV} & 1.6 & 2,112 & 1,686 & 3,628 & 2,695 & 1,716 & 1 \\
\hline
Thesaurus for Economics & \textbf{STW} & 8.06 & 6,524 & 6,524 & 31,189 & 57,907 & 6,531 & 1 \\
\bottomrule
\end{tabular}

}
\end{center}
\end{table}

\subsection{Results}

The results of this analysis are reported in Table~\ref{tab:results}, which shows the absolute number of possibly affected resources for each quality checking function and vocabulary combination. The value 0 means that we haven't found any affected resource, N/A means that a certain function was not applicable for reasons we explain below. An asterisk after certain numeric values indicates that due to performance reason, only an extrapolated subset containing 5\% of the respective elements (HTTP URIs or concepts) of the vocabulary has been analyzed.

\begin{table}[h]
\label{tab:results}
\caption{Results of the quality checking functions}

\begin{center}
\resizebox{\textwidth}{!} {
\setlength{\extrarowheight}{5pt}

\begin{tabular}{p{4cm}ccccccccccccccc}
\textbf{Issue} & \rotatebox{90}{\textbf{GTAA}} & \rotatebox{90}{\textbf{Geonames}} & \rotatebox{90}{\textbf{MeSH}} & \rotatebox{90}{\textbf{PXV}} & \rotatebox{90}{\textbf{Eurovoc}} & \rotatebox{90}{\textbf{IPSV}} & \rotatebox{90}{\textbf{Agrovoc}} & \rotatebox{90}{\textbf{DBpedia}} & \rotatebox{90}{\textbf{Pressinfo}} & \rotatebox{90}{\textbf{NYTP}} & \rotatebox{90}{\textbf{LCSH}} & \rotatebox{90}{\textbf{Meketre}} & \rotatebox{90}{\textbf{STW}} & \rotatebox{90}{\textbf{NAICS}} & \rotatebox{90}{\textbf{LVAk}} \\
\toprule
Omitted or Invalid Language Tags & 0 & 0 & 23,950 & 1,578 & n/a & 0 & 0 & 0 & 1,224 & 0 & 18 & 0 & 2 & n/a & 13,411 \\

Incomplete Language Coverage & 0 & 0 & 0 & 0 & n/a & 0 & 32,035 & 0 & 0 & 0 & 0 & 420 & 6,456 & n/a & 0 \\

Undocumented Concepts & 96,850 & 0 & 1,807 & 1,918 & 5,341 & 4,551 & 32,035 & 743,410 & 1,125 & 4,094 & 398,036 & 422 & 5,236 & 3,259 & 13,411 \\

Label Conflicts & 12,404 & 18 & 0 & 7 & n/a & 0 & 2,949 & 0 & 0 & 0 & n/a & 4 & 5 & n/a & 13 \\

\midrule

Orphan Concepts & 162,000 & 671 & 0 & 2 & 7 & 0 & 0 & 77,062 & 1,125 & 4,979 & 172,364 & 0 & 4 & 0 & 21 \\

Weakly Connected Components & 621 & 0 & 4 & 10 & 4 & 1 & 4 & 1,506 & 0 & 0 & 22,131 & 5 & 1 & 1 & 11 \\

Cyclic Hierarchical Relations & 0 & 0 & 4 & 0 & 0 & 0 & 0 & 1,132 & 0 & 0 & 0 & 0 & 0 & 0 & 5 \\

Valueless Associative Relations & 9,438 & 0 & 495 & 0 & 1 & 239 & 282 & 8,120 & 0 & 0 & 1,879 & 0 & 5,082 & 0 & 5 \\

Solely Transitively Related Concepts & 0 & 0 & 0 & 0 & 2,652 & 0 & 0 & 0 & 0 & 0 & 0 & 36 & 0 & 2,189 & 0 \\

Omitted Top Concepts & 9 & 9 & 0 & 0 & 1 & 0 & 0 & 0 & 0 & 1 & 18 & 0 & 0 & 0 & 0 \\

Top Concept Having Broader Concepts & 0 & 0 & 0 & 1 & 0 &0 & 0 & 0 & 0 & 0 & 0 & 0 & 0 & 0 & 0 \\

\midrule

Missing In-Links & 171,980* & 19 & 24625 & 1,686 & 6,796 & 3,080 & 32,035 & 733,800* & 1,125 & 20 & 404,540* & 422 & 6,516 & 2,213 & 13,411 \\

Missing Out-Links & 171,991 & 671 & 24626 & 1,472 & 6,797 & 0 & 32,035 & 743,410 & 1,116 & 0 & 408,198 & 273 & 6,524 & 0 & 13,411 \\

Broken Links & 0 & 0 & 1 & 163 && 1 & 238 & 0* & 11 & 7 && 425 & 1 & 3,169 & n/a \\


Undefined SKOS Resources & 0 & 0 & 1 & 0 & 0 & 1 & 0 & 0 & 0 & 0 & 0 & 0 & 0 & 0 & 0  \\

\bottomrule
\end{tabular}
}
\end{center}
\end{table}

% Labeling and Documentation Issues
We found labeling and documentation issues in all analyzed vocabularies. In nearly half of the evaluated vocabularies, language tags are used properly while others show a significant number of untaged resources. Incomplete Language Coverage were issues in three analyzed vocabularies, with e.g., AGROVOC supporting in total 25 languages but none of the concepts is documented in all of them. Most vocabularies didn't document their concepts with the documentary properties defined by SKOS. Label conflicts were also frequently observed, hence \todo{CM}{explain briefly} were found in several vocabularies: GTAA has many because? PXV has 7. Why given examples?

%\todo{BH}{this sentence is not clear, I assume you mean language tags, not labels.I'm not sure if relating language tags to concepts makes sense because a concept can be connected to multiple labels}PXV provides labels for XX\% of its concepts, Pressinfo labels only for 1 out of 1125 concepts, and the STW thesaurus misses labels for two concepts. Eurovoc and NAICS use SKOS-XL labels, which are currently not supported by qSKOS. Incomplete Language Coverage were issues in the three analyzed vocabularies: in AGROVOC all concepts had incomplete coverage, probably because \todo{CM}{es gibt labels in insgesamt 25 versch sprachen und keines der konzepte ist in jeder sprache beschrieben}, and in STW we found XX\%.


% Structural Issues
It could be observed that vocabularies that contain person or location names (GTAA and Geonames) have more orphan concepts than other vocabularies, establishing a network of terms in a specific domain (e.g., STW). Regarding weakly connected components, we see that those vocabularies that only consist of orphan concepts, contain no weakly connected components. Three vocabularies have one weakly connected component which means that there is one ``giant component''. All other vocabularies split into several ``islands'' of semantically related concepts. Hierarchical cycles are observed not to be a common issue, however they affect DBpedia to a great extent because for some reason many of the concepts are related broader to themselves. Valueless associative relations occur in 9 vocabularies, with their total number being relatively low compared to the total number of all semantic relations in the respective vocabularies. GTAA, Geonames, and NYTP have no top concept defined for any of the concept schemes they use. Eurovoc uses 128 concept schemes but only one without top concept. The opposite situation can be observed for LCSH which uses 19 concept schemes, but only one has a top concept. Only PXV vocabulary is affected by top concepts having broader concept in the current version. In an earlier version a different concept causing this issue could be spotted. According to the vocabulary creator it was an abaondoned concept, still available in the triplestore but maybe introduced by some bug in the vocabulary management software.

% Linked Data Specific Issues
With exception of NYTP and geonames we could observe that authoritative concepts of most vocabularies do not serve at link targets for other vocabularies. A slightly better situation can be found for missing out-links where most vocabularies also don't provide links to other vocabularies on the Web. There are three excepetions:  NYTP, IPSV and NAICS. Broken links are a very common issue in nearly every vocabulary. In the case of the Meketre vocabulary we could observe server-side misconfiguration as the cause. Checking this issue turned out difficult, taking into account the high number of links (over 400,000 in Eurovoc and LCSH) and our algorithm that sleeps three seconds after dereferencing each link in order to avoid an unintended DOS attack. So we decided to only check a subset of all links and extrapolate the results. Undefined SKOS resources seem to be a minor issue, we could only spot two of them in all vocabularies. 

% The \textbf{Loose Concepts} metrics provides insights on the structural complexity of vocabularies. The large number of loose concepts in GTAA can be explained by the fact that it contains names (e.g., persons or places) that arent't connected using skos properties. This is also the case for Geonames, NYTP and Pressinfo where the number of total conceps is equal to the number of loose concepts, indicating that no concept is semantically related to another concept.
% 
% The large number of \textbf{Weakly Connected Components} in GTAA is caused by many ``minimal'' components, containing only 2 concepts connected by a \texttt{skos:related} property. Due to the fact that loose concepts are not counted as weakly connected components, Geonames, NYTP and Pressinfo evaluate to zero components.
