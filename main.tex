\documentclass{llncs}

\usepackage{url}

\usepackage{verbatim}
\usepackage{amssymb}
\usepackage{rotating}
\usepackage{booktabs}

\usepackage{color}

\newcommand{\todo}[2]{\textbf{\textcolor{red}{(TODO [#1]: #2)}}}

% metrics = quantitative indicator of sth

\title{Quality Criteria for SKOS Vocabularies}
\author{Christian Mader\inst{1} \and Bernhard Haslhofer\inst{2}}
\institute{
	University of Vienna, Faculty of Computer Science\\\email{christian.mader@univie.ac.at}
	\and Cornell University, Information Science\\\email{bernhard.haslhofer@cornell.edu}}

\begin{document}

\maketitle

\begin{abstract}
    
The Simple Knowledge Organization System (SKOS) has become a standard model for controlled Web vocabularies. However, SKOS vocabularies are often heterogeneous in terms of quality, which reduces their applicability across system boundaries. Here we investigate how we can support taxonomists in judging the quality of SKOS vocabularies by quality criteria that go beyond the low-level data model integrity constraints defined in the SKOS specification. We identified existing \textbf{quality criteria}, formalized them into computable \textbf{quality checking functions} that identify affected resources in a given SKOS vocabulary. We implemented these functions in the qSKOS quality assessment tool, analyzed \todo{CM}{XX} existing vocabularies and found possible quality issues in \todo{CM}{YY} vocabularies.

\end{abstract}

%!TEX root = main.tex

\section{Introduction}\label{sec:introduction}

% What is SKOS?

The Simple Knowledge Organization System (SKOS) \todo{CM}{reference, no footnote} is a standard model for sharing and linking controlled vocabularies on the Web. Many organizations, including the European Union\footnote{EuroVoc, \url{http://eurovoc.europa.eu/}}, the United Nations\footnote{AGROVOC, \url{http://aims.fao.org/website/AGROVOC-Thesaurus/sub}}, or the UK government\footnote{Integrated Public Sector Vocabulary (IPSV), \url{http://doc.esd.org.uk/IPSV}} have created SKOS representations of their controlled vocabularies and published them on the Web so that they can easily be accessed and by humans and machines.

% What does "quality" mean in this context?

Each vocabulary reflects domain- and application-specific needs and vocabulary designers usually apply some guidelines or rules (cf.,~\cite{Coronado2009}) to meet certain quality requirements. However, these requirements can differ across domains and lead to quality problems when a vocabulary is used in other environments. As a consequence, currently available vocabularies are, even though they are expressed in SKOS, often heterogeneous in terms of quality. The following examples illustrate that this can affect their applicability for tasks like query expansion, faceted browsing, clustering, or auto-complete suggestions. 

\begin{itemize}

	\item The SKOS version of Agrovoc defines concepts in 25 different languages. However, almost all concepts have an English description, 38\% of all concepts don’t have German labels attached. This can be a problem for applications that use Agrovoc and rely on German translations.

	\item The previous version of the STW thesaurus, contained 5 pairs of concepts with identical labels. As a result, the auto-complete function of the online search interface suggested identical entries, with no disambiguation information.

	\item The vocabulary developed by the Austrian Armed Forces (LVAk thesaurus\footnote{Non-public thesaurus developed by the Austrian Armed Forces}), which contains 11 disconnected concept clusters. Confronting the thesaurus maintainers with these structures, they recognized them as “forgotten” test data that has no practical significance.

    % \item In the DBpedia category\footnote{\url{http://downloads.dbpedia.org/3.7/en/skos_categories_en.nt.bz2}} hierarchy, more than 10 percent of the categories not linked to any other category. When other vocabularies link to the DBPedia categories, e.g., to enable query expansion, it can negatively affect recall and reduce navigability.

\end{itemize}

% How can we improve quality?

%When a machine agent finds a vocabulary on the Web, it can check whether or not a vocabulary fulfills the low-level integrity conditions demanded by the SKOS specification. However, these
One reason for these quality problems is that SKOS currently only defines a set of \textbf{low-level integrity conditions}, which, when being enforced, ensure the structural and semantic requirements of the SKOS specifications but fail to capture \textbf{higher-level quality issues}, as those mentioned in the examples before. Another problem is that existing vocabulary design guidelines and best practices purely rely on \textbf{human assessment}, which is highly subjective and doesn’t scale when vocabulary quality assessment should be performed on larger, possibly linked Web vocabularies.

We believe, and recent developments of quality checking tools (e.g., PoolParty Thesaurus Consistency Checker\footnote{\url{http://demo.semantic-web.at:8080/SkosServices/check}}, qSKOS\footnote{\url{https://github.com/cmader/qSKOS}}, SKOSify\footnote{\url{http://code.google.com/p/skosify/}}) support this motivation. This, however, requires a formalization of the notion of quality with respect to SKOS vocabularies, which can be used in order to consistently implement quality assessment procedures across tools. Since, the notion of “quality” with respect to SKOS vocabularies naturally varies across application domains it is hard, if not impossible, to come up with a general definition of quality. However, we believe that vocabulary designers can apply a set of \textbf{formal quality criteria} on given vocabularies in order to assess whether or not a vocabulary fulfills their requirements. Since these criteria are based on a formal definition, they can be calculated by automated tools, which helps vocabulary designers in assessing the quality of available or newly created vocabularies and provide important feedback to the overall vocabulary design process.

The main contributions of this work can be summarized as follows:
\begin{itemize}
	\item We derived a \textbf{quality criteria catalogue} from existing literature, tools, and public mailing list discussions in the community. We \textbf{formalized} each criterion and implemented them in the \textbf{qSKOS quality assessment tool}.

	\item We \textbf{analyzed} a representative \textbf{set of existing SKOS vocabularies} to learn about possible quality issues in existing Web vocabularies.
\end{itemize}

The results from our studies show that... TBD


%!TEX root = main.tex

\section{Background}\label{sec:background}

% About Quality in SKOS vocabularies

SKOS is a language for defining vocabularies in the Web of Data and therefore based on the Open World Assumption. Established quality notions from closed-world systems, such as referential integrity or schema validation, don't hold anymore, because available information may be incomplete and non-explicitly stated facts cannot be determined as true or false. While trust and provenance models for Web data are being developed~\cite{Omitola2011,Hartig2009}, content-based and hand-crafted heuristics are currently used to evaluate quality in Linked Data sets~\cite{Heath2011}.

% MOVE TO RELATED WORK: Gueret et al. [\todo{CM}{find the right reference in google scholar}], for instance, calculate network measures to support human judgement on the quality of Linked Data \todo{CM}{?} graphs.


%In the course of the LATC project, Gueret et al. propose an infrastructure that utilizes crowd-sourcing technology to semi-automatically create links between datasets of the LOD cloud. To ensure the quality of the resulting network, human judgement is supported by calculating a report of locally approximated network measures. \todo{BH}{ich denke antoine hat das paper gemeint: http://www.mendeley.com/download/public/18928/4084690763/56bdde328dfbba2cb8594b57152c0e132ef00a40/dl.pdf . Leider fehlen da etliche Details um sich ein konkretes Bild machen zu können}

% SKOS Integrity Conditions (http://www.w3.org/TR/skos-reference/):
% S9    skos:ConceptScheme is disjoint with skos:Concept.
% S13   skos:prefLabel, skos:altLabel and skos:hiddenLabel are pairwise disjoint properties. (* NOT expressed formally *)
% S14   A resource has no more than one value of skos:prefLabel per language tag. (* NOT expressed formally *)
% S27   skos:related is disjoint with the property skos:broaderTransitive. (* NOT expressed formally *)
% S37   skos:Collection is disjoint with each of skos:Concept and skos:ConceptScheme.
% S46   skos:exactMatch is disjoint with each of the properties skos:broadMatch and skos:relatedMatch. (* NOT expressed formally *)
% -> 2 of 6 are expressed formally in OWL; the others cannot be expressed because OWL doesn't support disjoint properties

% Poolparty quality checker implementation:
% S13, S14, S27, S46 plus the formally defined S9, S37
% syntactic checks: URI, character encoding, whitespaces
% missing language tags: in SKOS labels and textual content)
% missing labels: prefLabels for skos:Concepts and rdfs:labels for skos:ConceptSchemes)
% loose concepts: concepts that are no top concepts and have no broaders

% SKOSify
% S13, S14, S27

% Mapping qSKOS Integrity Conditions <-> qSKOS quality criteria
% S13 + S14 combined in "Ambigious labels"
% S27 + S46 combined in "Associative vs. Hierarchical Relation Clashes" and "Exact vs. Associative and Hierarchical Mapping Clashes"

Interoperability between applications has been the major motivation for specifying in total six \textbf{SKOS integrity conditions}~\cite{Miles2005}, each of which is a statement that defines under which circumstances data are consistent with the SKOS data model. An example condition is ``a resource has no more than one value of \texttt{skos:prefLabel} per language tag''. Tools that can check whether these integrity conditions are met for given data are already available: two of the six integrity conditions are defined formally in the OWL representation of SKOS and can therefore be validated by any OWL reasoner. For validating a SKOS vocabulary against the integrity conditions, one can use tools such as the online \textbf{PoolParty Thesaurus Consistency Checker}, or the \textbf{Skosify} command line utility, which can validate SKOS vocabularies and also correct some detected quality problems.

% Argue why we need criteria that go beyond low-level integrity constraints and why we need to formalize them

Typical application areas of controlled vocabularies are classification, indexing, autocompletion, query reformulation/expansion, or serving as a glossary. As we discussed in detail in our earlier work \cite{Nagy2011}, these application areas have specific requirements with respect to vocabulary features, such as structure, availability, and documentation of conceptual resources. A vocabulary that doesn't fulfill these requirements can have a negative effect on an application's output and would therefore be judged by the application designers and potentially also by users as being a low-quality vocabulary. Missing relations between conceptual resources, for instance, can reduce retrieval recall when a vocabulary is used for query expansion. Missing entry points to concepts in a vocabulary can impede orientation for human users. Proprietary term definitions can hinder queries across distinct datasets, to name some examples.


% Argue why we need to formalize them -> because otherwise they will be implemented inconsistently across applications
Thus, in order to help vocabulary and application designers in judging the quality of SKOS vocabularies, we need higher-level quality criteria that go beyond low-level integrity conditions defining data model consistency. Such quality criteria are already well-established and have extensively been discussed in standardized guidelines \cite{ISO25964-1:2011,Z39.19:2005}, manuals \cite{Svenonius2003,Hedden2010,Aitchison2000,Harpring2010}, and scholarly articles \cite{Coronado2009,Soergel1997,Soergel2002}. 

%Whilst aformentioned literature also deals with online vocabulary publication and interoperability \cite{ISO25964-1:2011,Hedden2010}, covering requirements for publication on the Web of Data (as in, e.g., \cite{Heath2011,Hogan2010,Allemang2011}) goes beyond their scope. 
%On the other hand, publications targeting dataset and ontology development for the Web of Data don't elaborate on implications of the design decisions for controlled vocabularies. Our work seeks to narrow this gap by...

% TODO: bitte hier einfach nur die referenzen auf die wir uns beziehen zusammenfassen und kategorisieren. Schau dir dazu auch nochmals die SKOSify referenzen und die aus eurem iSemantics paper an.

If we want check existing SKOS vocabularies against these criteria, we first need to identify them and determine how to evaluate them given the syntax, structure and semantics of SKOS. Since we want to provide tool support for quality checking, we must formalize these criteria into computable quality checking functions, in order to ensure consistent implementation and interpretation across tools.

% I think for the big picture it is not necessary to go into such details. We can do this later when introducting the metrics

% The PoolParty Thesaurus Consistency Checker implements all these conditions and introduces the following new checks: \textbf{URI validation} checks for invalid characters like, e.g. whitespaces in the URI string. \textbf{Definition of labels} is tested for conceptual resources and concept schemes. Furthermore, SKOS labels and textual content is checked for \textbf{missing language tags} and the notion of \textbf{loose concepts} is defined: it encompasses conceptual resources that are no top concepts and have no broader concept.
% 
% The SKOSify tool implements three of the above mentioned SKOS integrity conditions as well as the \textbf{missing language tags} and \textbf{loose concepts} measures introduced by the PoolParty checker. It introduces \textbf{hierarchy cycle detection} and detection of \textbf{extra whitespace} for SKOS label or documentation properties.


%!TEX root = main.tex

\section{Quality Criteria for SKOS Vocabularies}\label{sec:criteria}

% Briefly describe methodology

We identified quality criteria in existing literature and manually checking existing vocabularies against these criteria. We published our preliminary results online\footnote{\url{https://github.com/cmader/qSKOS/wiki/Quality-Criteria-for-SKOS-Vocabularies}} and requested feedback from experts via public mailing list and informal face to face discussions. Based on the responses we received, we selected a subset \todo{CM}{due to lack of space, we cannot present all criteria. Do we have a good argument, why we present only a subset here? What were our selection criteria?} and formalized them into computable quality filtering function, which, in combination, can serve as quantitative quality indicators for SKOS vocabularies. Since each function operates on a given SKOS vocabulary, they are based on the following SKOS data model definition.


\begin{definition}[SKOS Vocabulary] Let the fully entailed RDFS interpretation~\cite{RDFSEM2012} (enriched by entailment of \texttt{owl:inverseOf} properties) of a SKOS vocabulary be a tuple of the form $V = \langle IR, C, SR, LV \rangle$, with
    
%    Based on the RDFS Interpretation $\mathcal{I}$ we define a vocabulary $V = (IR, C, A, LL, LR, SR, AR)$ (cf.~\cite{Hitzler2009}). V is expected to be fully entailed, based on the deduction rules for RDFS entailment defined in \cite{RDFSEM2012}. Let

\begin{itemize}
	\item \(IR = I_{CEXT}(\texttt{rdfs:Resource}^\mathcal{I})\) being the set of resources,

	\item $LV = I_{CEXT}(\texttt{rdfs:Literal}^\mathcal{I})$ being the set of literal values,

	\item \(C = I_{CEXT}(\texttt{skos:Concept}^\mathcal{I})\) being the set of \textbf{conceptual resources},

%\item \(A = I_{CEXT}(\texttt{skos:ConceptScheme}^\mathcal{I}) \cup I_{CEXT}(\texttt{skos:Collection}^\mathcal{I})\) be the set of \textbf{aggregation and grouping resources}.

%	\item LL being the set of \textbf{lexical labels}, which are instances of RDF plain literals.
%	\item \(LR \subseteq C \times LL\) being the set of \textbf{lexical relations} associating conceptual resources with lexical labels, e.g., via \texttt{rdfs:label} or subproperties thereof.

	\item \(SR = I_{EXT}(\texttt{skos:semanticRelation}^\mathcal{I})\) being the set of \textbf{semantic relations} associating conceptual resources with one another.

%\item \(AR \subseteq C \times A\) being the set of \textbf{aggregation relations} associating conceptual resources with instances of aggregation and grouping resources, hence instances of, e.g., \texttt{skos:inScheme} or \texttt{skos:member} or subproperties thereof.
\end{itemize}

\end{definition}


The quality criteria we identified have in common that they are applied on a given SKOS vocabulary and identify a set of resources, which do not fulfill that criterion. Therefore, we can regard a quality criterion as being an abstract filtering function $f:V \rightarrow \mathfrak{P}(IR)$, mapping a given vocabulary to the powerset of its resources. Each quality criterion we found can be implemented as a realization of such an abstract filtering function. Since the return value is a set of resources, statistical analyses can be computed on the number of resources in that set.

% Structure for each metric description
%- origin (refs) and design rationale
%- formal definition that specifies how to compute the metric

In the following, we describe the origins and design rationale for each metric and provide a semi-formal definition that specifies how to implement them in tools. We categorized the metrics \todo{CM,BH}{Briefly describe how we organize/structure the criteria}

% LEXICAL AND DOCUMENTATION CRITERIA

\subsection{Labeling and Documentation Criteria}
The criteria defined in this section primarily focus on supporting interaction with human users that either work with the vocabulary itself or interact with a system using the vocabulary for, e.g., search and retrieval purposes.

\subsubsection{Omitted Language Tags}
Some controlled vocabularies contain literals in natural language, but without information what language has actually been used. Using private language tags (see RFC3066) or omitting them at all limits, e.g., language-dependent queries.

\subsubsection{Incomplete Language Coverage}
As stated above, vocabularies should combine plain literals with language tags. Furthermore, to support, e.g., internationalization or translation use cases, every concept should have defined literal values in the same set of languages than the other concepts.

\subsubsection{Undocumented Concepts}
\cite{Aitchison2000} cites Svenonius to advocate ``inclusion of as much definition material as possible''. The SKOS language specification defines various properties that are intended to hold this kind of information, subsumed as ``documentation properties'' in the SKOS reference.

\subsubsection{Potentially Semantically Related Concepts} \todo{All}{reformulate criterion title}
This criterion is an extension of a statement from the SKOS primer, recommending ``no two concepts have the same preferred lexical label in a given language when they belong to the same concept scheme''. Given the fact that (i) many vocabularies don’t make use of concept schemes and (ii) also identical altLabels or hiddenLabels can have a negative effect in some use-cases (e.g, auto-completion of concept labels based on user input), we follow a more general approach: identification of all pairs of concepts with their respective prefLabel, altLabel or hiddenLabel meeting a threshold of a certain similarity function $sim:LV \times LV \rightarrow [0,1]$. For the analysis of the criterion in this document, we define the similarity function  to check for case-insensitive string equality and the threshold to equal 1.

% STRUCTURAL METRICS

\subsection{Structural Criteria}

The commonality of the following metrics is that they identify possible weaknesses when $V$ is used in scenarios exploiting the structure of its RDF graph. 

\subsubsection{Loose Concepts} is a criterion motivated by the notion of ``orphan terms'' in the literature~\cite{Hedden2010}, which are are terms without any associative or hierarchical relationships. Checking for them is common in existing thesaurus development applications and also suggested by \cite{Z39.19:2005}. When we apply this to SKOS vocabularies, a loose concept is a concept that has no semantic relations with other conceptual resources and thus do not occur in any of the pairs yielded by $SR$. Although they might have attached lexical labels, they lack valuable context information that is considered essential for, e.g., search query expansion. 

% \begin{definition}
% Let \(G_{lc} = (C, SR)\) and \(deg(c)\) be the degree, hence the number of in- and outgoing edges, of a node $c \in C$ in the graph $G_{lc}$. We can then define loose concepts as a function \(lc : G \rightarrow \mathbb{N}_{0}\), with \[lc(g) = \left|\left\{c \in C : deg(c) = 0\right\}\right|\]
% \end{definition}

\subsubsection{Weakly Connected Components} indicate that the vocabulary is split into separate ``clusters''. Presence of several WCC might be caused by incomplete data acquisition, ``forgotten'' test data, outdated terms, accidental deletion of relations and the like. In a practical setting, existence of weakly connected components could render the vocabulary less suitable for operations that rely on navigating a connected vocabulary structure, such as query expansion or suggestion of related terms. To calculate WCC, we create an undirected graph whose set of nodes constitutes all non-loose concepts and whose edges are defined by $SR$. We then utilize ``Tarjan’s algorithm''\cite{Hopcroft1973} that finds all connected components of the graph, i.e. all sets of conceptual resources that can reach each other by a path of semantic relations.

% \begin{definition}
% To evaluate this criterion we consider a graph \(G_{wcc} = (C - lc(G_{lc}), SR)\). The function \(components(g) : G \rightarrow  \mathfrak{P}(G)\) calculates all weakly connected components of a graph by replacing all directed edges by undirected edges and utilizing “Tarjan’s algorithm” \cite{Hopcroft1973}. Based on that function, we can express this criterion as a function \(wcc : G \rightarrow \mathbb{N}_{0}\), with \[wcc(g) = \left|components(g)\right|\]
% \end{definition}
 

\subsubsection{Cyclic Hierarchical Relations} have been mentioned numerous times in thesaurus development literature. \cite{Soergel2002} suggests a ``check for hierarchy cycles" since they ``throw the program for a loop in the generation of a complete hierarchical structure''. However, SKOS does not define a formal constraint regarding hierarchy cycles but the SKOS reference document mentions them as potential problems. The SKOS documentation also does not state, how exactly hierarchical relationships are to be interpreted. There exist common forms like, e.g., ``generic-specific'', ``instance-of'' or ``whole-part'' \cite{Hedden2010,Harpring2010,Aitchison2000} where cycles would be considered a logical contradiction. To calculate this criterion, we construct two graphs, with the set of nodes being $C$. The edges in one graph only point towards the broader concepts, the edges of the other graph towards the narrower concepts. For each graph we now identify those nodes that are part of a cycle and return the set of strongly connected components they are contained in.

%Erkärung: ein knoten (konzept) liegt oft in vielen überlappenden kreisen und es würde keinen sinn machen, diese alle auszugeben. deshalb berechne ich mir von jedem knoten der in einem kreis liegt, in welcher starken zusammenhangskomponente er liegt. das sind dann per definition alle knoten, die mit irgendeinem kreis in dem das konzept liegt was zu tun haben, weil eine starke zusammenhangskomponente ja eigentlich nur aus kreisen besteht (von jedem knoten gibt es einen gerichteten weg zu jedem anderen knoten)
 
% \begin{definition}
% Let \(H_{br} \subseteq SR\) be the set of pairs of hierarchically broader related resources, i.e. \(I_{EXT}(\texttt{skos:broader}^\mathcal{I}) \cup I_{EXT}(\texttt{skos:broaderTransitive}^\mathcal{I}) \cup I_{EXT}(\texttt{skos:broadMatch}^\mathcal{I})\). \(H_{br}\) also includes the extensions of the respecitive inverse properties with each pair in reverse order. Likewise, \(H_{nar} \subseteq SR\) can be defined as the set of hierarchically narrower related resources. We consider two graphs \(G_{br} = (C,H_{br})\) and \(G_{nar}=(C,H_{nar})\) and a function \(cycleNodes:G \rightarrow \mathfrak{P}(C)\) identifying all nodes in a graph, that are part of a cycle. Furthermore we consider a function \(stronglyConnectedSets: G \rightarrow \mathfrak{P}(C)\) that calculates all sets of nodes contained in a strongly connected subgraph of $G$. This criterion can be defined as a function \(chr:G \rightarrow \mathbb{N}_{0}\) with \[chr(g)=\left|\left\{s \in stronglyConnectedSets(g) : cn \in cycleNodes(g), cn \in s\right\}\right|\]
% \end{definition}


\subsubsection{Valueless Associative Relations}
The ISO/DIS 25964-1 standard suggests that terms that share a common broader term but don’t have an overlapping meaning, should not be related associatively. This is also advocated by \cite{Hedden2010} and \cite{Aitchison2000} who also mentions ``the risk that thesaurus compilers may overload the thesaurus with valueless relationships'', having a negative effect on precision. This criterion identifies a set of pairs of conceptual resources that share the same broader or narrower concept while also being associatively related, i.e. all pairs in $I_{EXT}(\texttt{skos:related}^\mathcal{I})$ that contain both of these conceptual resources.

\subsubsection{Solely Transitively Related Concepts}
Two concepts are related using only transitive hierarchical relations which are, according to the SKOS reference document, ``not used to make assertions''. Transitive hierarchical relations in SKOS are meant to be infered by the vocabulary user, which is reflected in the SKOS schema by, e.g., \texttt{broader} being a subproperty of \texttt{broaderTransitive}. This criterion identifies the set $I_{EXT}(\texttt{skos:broaderTransitive}^\mathcal{I}) \cup I_{EXT}(\texttt{skos:narrowerTransitive}^\mathcal{I})$ \textbf{without} RDFS subproperty entailment.

\subsubsection{Omitted Top Concepts}
The SKOS language provides ConceptSchemes which are a facility for grouping related concepts. In order to provide entry points to such a group of concepts, one or more concepts can be marked as top concepts. This helps to provide ``efficient access'' (SKOS primer) and simplifies orientation in the vocabulary. 

%Thus, this criterion returns the subset of $C$ whose elements are not contained in any of the pairs obtained by $I_{EXT}(\texttt{skos:hasTopConcept}^\mathcal{I})$.

\subsubsection{Top Concept Having Broader Concepts}
\cite{Allemang2011} proposes to ``not indicate any concepts internal to the tree as top concepts'', i. e. top concepts should not have broader concepts.

% LINKED-DATA SPECIFIC CRITERIA

\subsection{Linked Data Specific Criteria}

This section identifies criteria related to a vocabulary's interaction with other datasets in the LOD cloud.

\subsubsection{Missing In-Links}
A periodically updated version of a diagram visualizing ``the data sets in the LOD cloud as well as their interlinkage relationships'' is publised online\footnote{\url{http://www4.wiwiss.fu-berlin.de/lodcloud/state/}}. In these diagrams, dbpedia is shown as the most intensely linked resource, i.e. a large number of vocabularies link to resources in that dataset. Therefore a dataset with a large number of other datasets referencing it, can be assumed to be of great value for the community. Accordingly, estimating the number of In-Links of SKOS concepts defined in a vocabulary (henceforth called the in-degree of a concept) would give some hint about how established or accepted a controlled vocabulary is. A method to carry out these estimations would be utilizing existing Linked Data indices like, e.g., Sindice\footnote{\url{http://sindice.com/}} that provide a SPARQL endpoint. Accordingly, this criterion identifies the set of all conceptual resources that lack any In-Links.

\subsubsection{Missing Out-Links}
The Web of Data consists of globally interconnected datasets, ``enabling seamless connections between data sets''\cite{Heath2011}. These links can be established between various SKOS vocabularies by Out-Links, i.e., references from conceptual resources in the ``source vocabulary'' $V$ to resources identified by an HTTP URI and available at a different host. Similar to the criterion above, this criterion identifies the set of all conceptual resources that lack such Out-Links.

%\subsubsection{HTTP URI Schema Violation}\todo{All}{We don't have any evidence this really happens in existing vocabularies, although it is very likely. Maybe we should skip this.}

\subsubsection{Link Target Unavailability}
This criterion identifies a set of all resources that have been identified as unavailable, i.e. resolving their URI leads to an erroneous HTTP response or no response at all. An erroneous HTTP response in that case can be defined as a response code (after a possible redirection) other than 200. Just as in the ``traditional'' Web, these ``broken links'' hinder information gathering and should be avoided. 

% OTHER CRITERIA

\subsection{Other Criteria}

\subsubsection{Undefined SKOS Property Usage}
Some vocabularies reference resources in the SKOS namespace that cannot be resolved. A common reason for this is, that there seems to exist a misunderstanding about the correct way of introducing new classes or properties in an OWL document. In some cases, vocabularies ``invent'' new terms in the SKOS namespace in order to meet certain requirements of the publishing organizations that cannot (yet) be expressed within the SKOS schema. Another reason for the presence of unresolvable resource references in the SKOS namespace is that the vocabulary might be outdated. With the SKOS W3C recommendation maturing, some properties have been removed from the current version of the schema. Vocabularies using these ``deprecated'' properties hence contain parts of their information in a non ``standard"-compliant way, having negative effects when, e.g., crawling vocabularies to gather specific information.



%!TEX root = main.tex

\section{Analysis of Existing SKOS Vocabularies}\label{sec:analysis}

\subsection{The qSKOS Quality Checking Tool}

\todo{BH}{make clear that tool also provides reports not only numeric values}

\subsection{Vocabulary Data Set}
Our vocabulary dataset covers XY vocabularies from various domains differing in statistical properties (cf. Table~\ref{vocabs}). The data was gathered in different ways. Some vocabularies where downloaded from their respective project websites or, in the cases of IPSV, Peroxisome Knowledge Base, and LVAk requested directly from the creators. The MeSH dataset was created by \cite{Assem2006} and is available for download\footnote{\url{http://thesauri.cs.vu.nl/eswc06/}}. We created the SKOS version of LVAk thesaurus from CSV files provided by the vocabulary maintainers. The vocabulary structure was uncomplex, so the conversion process to SKOS turned out as being rather straightforward. \todo{CM}{should we elaborate more on the conversion process?}
 
Our syntactical checks showed that only the AGROVOC vocabulary contained invalid datatypes and characters. After some minor syntax modifications, which don’t influence the evaluation against our criteria, we were also able to include this vocabulary. We also informed the AGROVOC team about this problem, which will try to fix it.


\begin{table}[h]
\caption{Analyzed vocabularies}
\begin{tabular}{lp{5cm}cccccc}
\textbf{Domain} & \textbf{Vocabulary} & \rotatebox{90}{\textbf{Concepts}} & \rotatebox{90}{\textbf{Lexical Rel.}} & \rotatebox{90}{\textbf{Semantic Rel.}} & \rotatebox{90}{\textbf{Aggregation Rel.}} & \rotatebox{90}{\textbf{Concept Schemes}} & \rotatebox{90}{\textbf{Collections}}\\
\toprule
Media & Gemeenschappelijke Thesaurus Audiovisuele Archieven (GTAA) & 171991 & 178776 & 50892 & 343980 & 9 & 0 \\
\hline
Science & Geonames Ontology & 671 & 671 & 0 & 671 & 9 & 0 \\
\hline
& Medical Subject Headings (MeSH) & 24626 & 150617 & 38858 & 0 & 0 & 0 \\
\hline
& Peroxisome Knowledge Base & 2112 & 3628 & 2695 & 1716 & 1 & 0 \\
\hline
Government & Eurovoc & 6797 & 457788 & 18491 & 15512 & 128 & 0 \\
\hline
& Integrated Public Sector Vocabulary (IPSV) & 4732 & 7945 & 13843 & 4483 & 3 & 0 \\
\hline
Publications & Agrovoc & 31993 & 572138 & 63936 & 32043 & 1 & 0 \\
\hline
& DBpedia Categories & 743410 & 740352 & 1490316 & 0 & 0 & 0 \\
\hline
& University of Southampton Pressinfo & 1125 & 0 & 0 & 0 & 0 & 0 \\
\hline
& New York Times People & 4979 & 4979 & 0 & 4979 & 1 & 0 \\
\hline
& Library of Congress Subject Headings (LCSH) & 459182 & 746076 & 595754 & 815816 & 19 & 0 \\
\hline
Cross-domain & Meketre & 422 & 569 & 1698 & 6 & 2 & 0 \\
\hline
Economy & Thesaurus for Economics (STW) & 6524 & 31189 & 57907 & 6531 & 1 & 0 \\
\hline
& North American Industry Classification System (NAICS) & 4175 & 0 & 8684 & 2235 & 1 & 0 \\
\hline
Military & Thesaurus Landesverteidigungsakademie (LVAk) & 13411 & 17250 & 16346 & 0 & 0 & 0 \\
\bottomrule
\end{tabular}
\label{vocabs}
\end{table}


\subsection{Results}

\todo{CM}{Add the result table, with one row for the lc metric.}

\todo{CM}{Add result interpretation for this one metric}


\section{Related Work}\label{sec:related_work}

The problem of ``vocabulary quality'' is closely related to the more general problem of “data quality”, which has intensively been discussed in data and information systems research. \cite{Pipino2002,Batini2009} argue that dealing with data quality should involve both “subjective perceptions of the individuals” and “objective measurements based on the data set in question”. We see our work as a contribution to the latter and believe that the results of SKOS quality measures must be combined with domain knowledge, and therefore human expertise, to make final quality judgements.

A vast number of publications in the field of designing, validating and testing controlled vocabularies is available. Many of them propose properties \cite{Soergel1995}, design guidelines \cite{Svenonius2003,...} or metrics \cite{Elkin2002,Kless2010} aiming to improve, e.g., retrieval precision and recall, consistency and multilinguality. The work is based on various sources like, e.g., review of existing vocabularies~\cite{Soergel1995} or survey-based studies~\cite{Pinto2008}. However, there are hardly any formally defined quality indicators that can be automatically assessed without further knowledge about the application domain, targeted user group or usage scenario.

%For controlled vocabularies, \cite{Svenonius2003} argues precision and recall to be “the chief objectives of any retrieval language” and proposes a set of design guidelines (e.g., term selection, structural and syntactical considerations) aiming to improve these objectives. Elkin and XXX? [Elkin2002#] identify general quality metrics and structural and maintainability considerations for controlled vocabularies in the health domain, which cover issues like non-redundancy, consistency, and multilinguality of concepts? Soergel [Soergel1995#] mentions properties of “good” thesauri like, e.g., support for synonym and hierarchic expansion or request-oriented indexing. As negative properties of the AAT Soergel points out, e.g., missing cross-references, incomplete facet analysis, the monohierarchical layout and shortcomings in term form and choice. On the positive side he mentions, e.g., presence of scope notes, definition of related terms and some definitions of synonyms. Pinto [Pinto2008#] performs a survey-based study comparing perceived and expected properties (“variables”) of thesauri used in social science databases. The findings exposed need for “considerable improvement” in the structural perspective (e.g., equivalence and associative relationships), but also regarding “performance” variables (e.g., explanatory notes or expected performance) and “format” issues (e.g., ergonomics and display). Some of the identified variables (e.g, pre-coordination, searching options) are hard do assess automatically or are application dependent, others aren’t exactly defined or formalized. However, neither of them suggest any automatic assessment methods, which can be applied to controlled Web vocabularies.

Research on the notion of “data quality” has also been conducted in Semantic Web research. For general datasets, \cite{Heath2011,Hogan2010} propose best practices and ..   \cite{Tartir2007} 

Hogan and ??? [Hogan2009#] identify four categories of common errors and shortcomings in RDF documents. Some of them are also of great importance and can be adopted for assessing the quality of SKOS vocabularies, especially considering interoperability issues like e.g., dereferencability and use of undefined classes and properties. However, the author’s focus lies on RDF datasets in general, thus features of controlled vocabularies are not covered. This is also the case for [Tartir2007#] who propose a framework for evaluation and ranking of Ontologies. Given the fact that vocabularies usually extend the SKOS schema only to a very limited degree by defining subclasses and subproperties, the metrics defined in [Tartir2007] are not considered applicable to these vocabularies.

[Arpinar2006#] define three types of conflicts that may occur in ontologies. Based on these types, the ontology maintainers can create rules in RuleML or SWRL to find violations in their ontologies. Although not using a rule language, we express quality criteria formally, following a very similar approach.

[Gangemi2005#] introduce three “measure types for ontology evaluation” (dimensions), i.e. structural, functional and usability-related measures. The structural exploits the graph structure of an ontology and defines a number of measure like, e.g., absolute, average or maximum depth and breadth, number of leaf nodes, number of siblings or density. While for many of the measures, its applicability for controlled vocabularies is not immediately clear, some (e.g., cycle ratio, inverse relations) resemble common patterns in controlled vocabularies and thus are represented in our catalog of quality criteria. Functional dimensions are measures related to how well an ontology fits its intended purpose like, e.g., recall, precision and accuracy. They are highly context-dependent, however, we expect most of our identified criteria to impact these dimensons. In the usability dimension, Gangemi et al. mention three analytical levels for usability profiling: recognition, efficiency, and interfacing. Since they don’t go into details on what the implications for thesauri are, we try to complement this in the rationale of our identified quality crtieria. [Brank2005#] provide a concise overview on ontology evaluation approaches
Linked Data Quality

[Yeganeh2011#] propose a method to convert semi-structured data (e.g., XML files) into “high-quality Linked Data”. Their notion of quality encompasses usage of HTTP URIs, URI dereferencability, linkage of related objects as well as merging and detection of duplicate resources.


%!TEX root = main.tex
\section{Conclusions and Future Work}\label{sec:conclusions}

In this paper we identified possible quality issues in SKOS vocabularies and implemented them as quality checking functions in our qSKOS quality assessment tool. We analyzed a representative set of existing SKOS vocabularies and found  issues in all of them. Labeling and documentation issues were omnipresent and also structural issues, which require further investigation by the vocabulary maintainers, were found in most vocabularies. Although SKOS is designed for Linked Data, many existing vocabularies still resemble their closed-system origin, which results in a relatively low number of in- and out-links. Broken links is a major issue and calls for synchronization mechanisms in order to maintain navigability between concepts.

During our work on qSKOS and while writing this paper we were in contact with some of the creators of the vocabularies we analyzed and also reported back initial results from our analysis. At the time of this writing, we know that our findings lead to improvements in at least two SKOS vocabularies.

We are aware that the quality issues we described in this work are purely quantitative indicators that, on their own, cannot be generalized into statements about the quality of a vocabulary. To learn more about the real-world impact of our work, we would like to conduct a qualitative follow-up study, in which we confront taxonomists with our results. We will also further collect community feedback, enhance the issues list, and set up a Web-based SKOS quality checking service. For larger vocabularies we will also increase the performance and efficiency of our quality checking algorithms.

\bibliography{references}
\bibliographystyle{splncs03}

\end{document}
