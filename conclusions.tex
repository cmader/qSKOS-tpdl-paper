%!TEX root = main.tex
\section{Conclusions and Future Work}\label{sec:conclusions}

In this paper we presented possible quality issues in SKOS vocabularies and described how we implemented them as quality checking functions in our qSKOS quality assessment tool. We analyzed a representative set of existing SKOS vocabularies and found  issues in all of them. Labeling and documentation issues were omnipresent and also structural issues, which require further investigation by the vocabulary maintainers, were found in most vocabularies. Although SKOS is designed for Linked Data, many existing vocabularies still resemble their closed-system origin, which results in a relatively low number of in- and out-links. Broken links are a major issue and call for synchronization mechanisms in order to maintain navigability between concepts in different vocabularies.

We are aware that these issues are purely quantitative quality indicators. To learn more about the real-world impact of our work like, e.g., the relative importance of the identified quality issues, we will conduct a qualitative follow-up study, in which we discuss these results with more taxonomists. We will also further collect community feedback, enhance the issues list, and set up a Web-based SKOS quality checking service.

We already reported initial results from our analysis to some of the maintainers of the vocabularies we analyzed. At the time of this writing, we know that our findings led to improvements in at least two SKOS vocabularies.
