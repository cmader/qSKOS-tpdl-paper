%!TEX root = main.tex

\section{Background and Definitions}

% About Data Quality and SKOS

SKOS is a language for defining vocabularies in the Web of Data and therefore based on the Open World Assumption. Established quality notions from closed-world systems, such as constraints, referential integrity or schema validation, don’t hold anymore, because the available information may be incomplete and facts that are not explicitly stated cannot be determined as true or false. While trust and provenance models for Web data are being developed \cite{Omitola2011,Hartig2009}, content-based and hand-crafted heuristics are currently used to evaluate quality in Linked Data sets \cite{Heath2011}.


% What kind of quality criteria are defined in SKOS

\todo{CM}{write a short paragraph about the quality criteria in the SKOS specification.}



% This definition is not a quality criterions, henec it is in the background section (we need to discuss that)

\begin{definition} \textbf{The SKOS Baseline Model} is defined as follows:
Let R be the set of all resources, as defined in [Web Architecture Document] and \[V = (C, A, LL, LR, SR, AR)\] be one specific SKOS vocabulary, with

\begin{itemize}
	\item \(C \subseteq R\) being the set of \textbf{conceptual resources} of type \texttt{skos:Concept}.
	\item \(A \subseteq R\) being the set of aggregation and grouping resources in SKOS, hence \texttt{skos:ConceptSchema} and \texttt{skos:Collection}.
	\item LL being the set \textbf{lexical labels}, which are instances of RDF plain literals.
	\item \(LR \subseteq C \times LL\) being the set of \textbf{lexical relations} associating conceptual resources with lexical labels, hence instances of \texttt{skos:prefLabel}, \texttt{skos:altLabel}, or \texttt{skos:hiddenLabel}.
\item \(SR \subseteq C \times C\) being the set of \textbf{semantic relations} associating conceptual resources with concepts, hence instances of \texttt{skos:semanticRelation} and subproperties thereof.
\item \(AR \subseteq C \times A\) being the set of \textbf{aggregation (and grouping) relations} associating conceptual resources with instances of aggregation and grouping resources.
\end{itemize}
\end{definition}
